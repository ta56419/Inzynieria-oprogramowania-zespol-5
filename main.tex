\documentclass[a4paper,12pt]{article}
\usepackage{geometry}
\usepackage{titlesec}
\usepackage{tocloft}
\usepackage{graphicx}
\usepackage{longtable}
\usepackage{booktabs}
\usepackage[T1]{fontenc}


\geometry{left=2.5cm, right=2.5cm, top=2.5cm, bottom=2.5cm}


\begin{document}
\begin{titlepage}
    \centering
    \vspace*{5cm}
    {\Huge \textbf{Dokumentacja Projektu Zaliczeniowego}}\\[1.5cm]
    {\LARGE Przedmiot: Inżynieria Oprogramowania}\\[1cm]
    {\Large Temat: \textbf{System zarządzania siłownią}}\\[1cm]
    {\Large Autorzy: \textbf{Andrii Tynskyi, Bohdan Shcherbina}}\\[1cm]
    {\Large Grupa: I5-220B}\\[1cm]
    {\Large Kierunek: Informatyka}\\[1cm]
    {\Large Rok akademicki: 2}\\[1cm]
    {\Large Poziom i semestr: I/4}\\[1cm]
    {\Large Tryb studiów: Stacjonarne}\\[1cm]
    \vfill
    {\large \today}
\end{titlepage}

% Należy pozostawić wszelkie nagłówki tego dokumentu, a umieszczać treść w odpowiednich miejscach zamiast obecnych objaśnień.
% Stronę tytułową można sformatować w dowolny sposób, ale należy pozostawić zawartość informacyjną w układzie pokazanym powyżej.
% Praca powinna zostać złożona wyłącznie w formacie pdf. Przed wygenerowaniem ostatecznej wersji należy zaktualizować spis treści – wyświetlane dwa poziomy.
% Niniejszą informację należy również usunąć z wersji końcowej.


\renewcommand{\contentsname}{Spis treści}
\tableofcontents
\newpage


\section{Odnośniki do innych źródeł}
GitHub: https://github.com/ta56419/Inzynieria-oprogramowania-zespol-5/branches .

System zarządzania siłownią to praktyczny projekt, który łączy wiele funkcjonalności, takich jak rejestracja członków, planowanie treningów i monitorowanie postępów, co pozwala na realne zastosowanie w biznesie. Branża fitness dynamicznie się rozwija, a cyfryzacja usług, takich jak rezerwacje zajęć grupowych czy płatności online, staje się coraz bardziej powszechna, co czyni ten projekt aktualnym i przyszłościowym. Temat siłowni jest interesujący, ponieważ obejmuje zarówno aspekty techniczne, jak i użytkowe, a system może być rozwijany o dodatkowe funkcje, takie jak integracja z aplikacjami mobilnymi czy analiza wyników treningowych.


\newpage

\section{Słownik pojęć}
Tabela lub lista z pojęciami, które wymagają wyjaśnienia, wraz z tymi wyjaśnieniami – w szczególności synonimy różnych pojęć używanych w dokumentacji.


\newpage

\section{Wprowadzenie}
\subsection{Cel dokumentacji}


\subsection{Przeznaczenie dokumentacji}
Dla kogo jest przeznaczona dokumentacja.

\subsection{Opis organizacji lub analiza rynku}
Jedna z dwóch opcji:

\begin{enumerate}
    \item Jeśli dla konkretnej organizacji: Czym jest organizacja, dla której realizowany będzie system; jak działa (lub będzie działała), kiedy system będzie wdrożony – tutaj nie odwołujemy się do samego systemu, tylko opisujemy samo działanie organizacji i role. W szczególności – jak wyglądają główne procesy biznesowe.
    \item Jeśli na masowy rynek: Pobieżna analiza rynku. Dla kogo będzie przydatny taki system. Ile jest organizacji, które będą mogły z niego skorzystać, użytkowników w poszczególnych organizacjach. Czy te organizacje stanowią jednorodną grupę czy są różne rodzaje. Co one mają ze sobą wspólnego. Jak ta liczba będzie się zmieniała w najbliższej przyszłości.
\end{enumerate}


\subsection{Analiza SWOT organizacji}
Tabela przedstawiająca mocne i słabe strony oraz szanse i zagrożenia.

\begin{itemize}
    \item jeśli system dla konkretnej organizacji:
    \begin{itemize}
        \item wystarczy sama tabela 2x2 (silne-słabe-szanse-zagrożenia)
    \end{itemize}
    \item jeśli system na masowy rynek:
    \begin{itemize}
        \item szanse i zagrożenia
    \end{itemize}
\end{itemize}

\newpage

\section{Specyfikacja wymagań}
\subsection{Charakterystyka ogólna}
Opis systemu i jego podstawowych założeń.
\subsubsection{Definicja produktu}
Jedno zdanie o systemie – nazwa i rodzaj.

\subsubsection{Podstawowe założenia}
Do czego będzie służył ten system? – Kilka/kilkanaście zdań wprowadzających.

\subsubsection{Cel biznesowy}
Co organizacja docelowa chce osiągnąć wdrażając system?

\subsubsection{Użytkownicy}

Lista – ew. wyjaśnienia dodać do słownika pojęć.

\subsubsection{Korzyści z systemu}
Dla poszczególnych grup użytkowników – każdy element z unikalnym numerem identyfikacyjnym.

\subsubsection{Ograniczenia projektowe i wdrożeniowe}

Przepisy prawne, specyficzne technologie, narzędzia, b.d., protokoły komunikacyjne, aspekty zabezpieczeń, zgodność ze standardami, powiązania z innymi aplikacjami, platforma sprzętowa, system operacyjny, inne komponenty niezbędne do współpracy – wszystko wraz z uzasadnieniem!


\subsection{Wymagania funkcjonalne}


\subsubsection{Lista wymagań}

Lista numerowana – czyli lista przypadków użycia lub bardziej ogólnie sformułowane wymagania, np. wymagania użytkownika.

\subsubsection{Diagramy przypadków użycia}
Tutaj same diagramy – bez specyfikacji, ale każdy diagram z tytułem i na osobnej stronie.

\subsubsection{Szczegółowy opis wymagań}

Dla 5-7 wybranych najważniejszych przypadków użycia – przypadku zespołów 3-osobowych i większych, proporcjonalnie więcej
każde na nowej stronie wg następujących punktów:

\begin{itemize}
    \item Numer – jako ID
    \item Nazwa
    \item Uzasadnienie biznesowe – odwołanie (-a) do elementów wymienionych w 4.1.5. (id i treść elementu, do którego się odwołujemy)
    \item Użytkownicy
    \item Scenariusze, dla każdego z nich:
    \begin{itemize}
        \item Nazwa scenariusza
        \item Warunki początkowe
        \item Przebieg działań – numerowana lista kroków, ze wskazaniem, kto realizuje dany krok
        \item Efekty – warunki końcowe
        \item Wymagania niefunkcjonalne – szczegółowe wobec poszczególnych wymagań funkcjonalnych
        \item Częstotliwość - na skali 1-5 lub BN-BW
        \item Istotność – inaczej: zależność krytyczna, znaczenie - na skali 1-5 lub BN-BW
    \end{itemize}
\end{itemize}

\textbf{Ważne!}

Elementy od warunków początkowych do końca mogą być grupowane, tj. specyfikacja pojedynczego przypadku użycia może zawierać:
- pojedynczy przebieg działań (scenariusz główny) oraz ew. scenariusze alternatywne, albo
- wiele przebiegów głównych wraz z ew. scenariuszami alternatywnymi – wtedy każdy z przebiegów głównych powinien być opisany wg tych punktów (od warunków początkowych do końca).


\subsection{Wymagania niefunkcjonalne}
W odniesieniu do całego systemu, modułów lub innych składowych systemu:
\begin{enumerate}
    \item Wydajność – w odniesieniu do konkretnych sytuacji – funkcji systemu
    \item Bezpieczeństwo – utrata, zniszczenie danych, zniszczenie innego systemu przez nasz – wraz z działaniami zapobiegawczymi i ograniczającymi skutki
    \item Zabezpieczenia
    \item Inne cechy jakości – najlepiej ilościowo, żeby można było zweryfikować (zmierzyć) – adaptowalność, dostępność, poprawność, elastyczność, łatwość konserwacji, przenośność, awaryjność, testowalność, użyteczność
\end{enumerate}

\newpage

\section{Zarządzanie projektem}
\subsection{Zasoby ludzkie}
(rzeczywiste lub hipotetyczne) – przy realizacji projektu
Należy założyć, że projekt byłby realizowany w całości jako projekt komercyjny a nie tylko częściowo w ramach zajęć na uczelni


\subsection{Harmonogram prac}
Etapy mogą się składać z zadań.
Wskazać czasy trwania poszczególnych etapów i zadań – wykres Gantta.
obejmuje również harmonogram wdrożenia projektu – np. szkolenie, rozruch, konfiguracja, serwis – może obejmować różne wydania (tj. o różnej funkcjonalności – personal, professional, enterprise) i wersje (1.0, 1.5, itd.)

\subsection{Etapy/kamienie milowe projektu}

Dla głównych etapów projektu.


Cały ten rozdział jest opcjonalny – dla chętnych. Nie jest omawiany na wykładzie!
Studenci powinni skonsultować szczegółowe wymagania w tym zakresie z nauczycielem prowadzącym zajęcia w danej grupie.

\newpage

\section{Zarządzanie ryzykiem}
\subsection{Lista czynników ryzyka}
Wypełniona lista kontrolna.

\subsection{Ocena ryzyka}
Prawdopodobieństwo i wpływ

\subsection{Plan reakcji na ryzyko}
Działania w odniesieniu do poszczególnych ryzyk.
Mogą być wg różnych strategii, tj. kilka strategii dla pojedynczego czynnika ryzyka 


Rozdział obowiązkowy w zespołach co najmniej 3-osobowych, w mniejszych – do uzgodnienia z prowadzącym zajęcia.

\newpage

\section{Zarządzanie jakością}
\subsection{Scenariusze i przypadki testowe}

Głównie testowanie funkcjonalności, ale może być też testowanie wymagań niefunkcjonalnych/zgodności; każdy scenariusz od nowej strony, musi zawierać co najmniej następujące informacje (sugerowany układ tabelaryczny, np. wg szablonu podanego w osobnym pliku lub na wykładzie):

\begin{itemize}
    \item numer – jako ID
    \item nazwa scenariusza – co test w nim testowane (max kilka wyrazów),
    \item kategoria – poziom/kategoria testów,
    \item opis – dodatkowe opcjonalne informacje, które nie zmieściły się w nazwie,
    \item tester - konkretna osoba lub klient/pracownik,
    \item termin – kiedy testowanie ma być przeprowadzane,
    \item narzędzia wspomagające – jeśli jakieś są używane przy danym scenariuszu,
    \item przebieg działań – tabela z trzema kolumnami: lp. oraz opisującymi działania testera i systemu,
    \item założenia, środowisko, warunki wstępne, dane wejściowe – przygotowanie przed uruchomieniem testów,
    \item zestaw danych testowych – najlepiej w formie tabelarycznej – jakie konkretnie dane mają być użyte przez testera i zwrócone przez system w poszczególnych krokach przebiegu działań
    \item \textit{przebieg lub zestaw danych testowych musi zawierać jawną informację o warunku zaliczenia testu}
\end{itemize}



\newpage

\section{Projekt techniczny}
\subsection{Opis architektury systemu}
Z ew. rysunkami pomocniczymi

\subsection{Technologie implementacji systemu}

Tabela z listą wykorzystanych technologii, każda z uzasadnieniem.

\subsection{Diagramy UML}

każdy diagram ma mieć tytuł oraz ma być na osobnej stronie
diagramy przypadków użycia umieszczone w punkcie 4.2.2, a nie tutaj.

\subsubsection{Diagram(-y) klas}
1 lub więcej

\subsubsection{Diagram(-y) czynności}
Co najmniej 1 dla zespołów 2-osobowych, więcej dla liczniejszych

\subsubsection{Diagramy sekwencji}
Co najmniej 5, w tym co najmniej 1 przypadek użycia zilustrowany kilkoma diagramami (dla zespołów 2-osobowych, dla liczniejszych więcej)

\subsubsection{Inne diagramy}
Co najmniej trzy – komponentów, rozmieszczenia, maszyny stanowej itp.

\subsection{Charakterystyka zastosowanych wzorców projektowych}

informacja opisowa wspomagana diagramami (odsyłaczami do diagramów UML); jeśli wykorzystano wzorce projektowe, to należy wykazać dwa z nich
uwaga – wzorce projektowe nie są omawiane na wykładach!

\subsection{Projekt bazy danych}

\subsubsection{Schemat}
w trzeciej formie normalnej; jeśli w innej to umieć uzasadnić wybór

\subsubsection{Projekty szczegółowe tabel}
w zależności, czy następujące elementy są widoczne na schemacie b.d.: nazwa tabeli, nazwy pól, typ danych, wartości NULL, klucz główny, klucz obcy –
- jeśli TAK: i nie ma potrzeby pokazania dodatkowych elementów b.d., to ten punkt może być pusty,
- jeśli NIE: to podać te elementy, których nie widać na schemacie.
dodatkowymi elementami mogą być np. triggery, procedury, funkcje, indeksy, użytkownicy, role. 

\subsection{Projekt interfejsu użytkownika}
Co najmniej dla głównej funkcjonalności programu – w razie wątpliwości, uzgodnić z prowadzącym zajęcia.

\subsubsection{Lista głównych elementów interfejsu}

Okien, stron, aktywności (Android)

\subsubsection{Przejścia między głównymi elementami}

Np. storyboard, schemat blokowy lub inna notacja

\subsubsection{Projekty szczegółowe poszczególnych elementów}
dla 5-7 głównych elementów (w zespołach 2-osobowych)
każdy element od nowej strony z następującą minimalną zawartością:

\begin{itemize}
    \item numer – ID elementu
    \item nazwa – np. formularz danych produktu
    \item projekt graficzny – wystarczy schemat w narzędziu graficznym lub zrzut ekranu – z przykładowymi danymi (nie pusty!!!)
    \item opcjonalnie:
    \item opis – dodatkowe opcjonalne informacje o przeznaczeniu, obsłudze – jeśli nazwa nie będzie wystarczająco czytelna
    \item wykorzystane dane – jakie dane z bazy danych są wykorzystywane
    \item opis działania – tabela pokazująca m.in. co się dzieje po kliknięciu przycisku, wybraniu opcji z menu itp.
\end{itemize}

\subsection{Procedura wdrożenia}
Jeśli informacje w harmonogramie nie są wystarczające (a zapewne nie są)

\newpage

\section{Dokumentacja dla użytkownika}
Opcjonalnie – dla chętnych
Na podstawie projektu docelowej aplikacji, a nie zaimplementowanego prototypu architektury

4-6 stron z obrazkami (np. zrzuty ekranowe, polecenia do wpisania na konsoli, itp.)
\begin{itemize}
    \item pisana językiem odpowiednim do grupy odbiorców – czyli najczęściej nie do informatyków
    \item może to być przebieg krok po kroku obsługi jednej głównej funkcji systemu, kilku mniejszych, instrukcja instalacji lub innej pomocniczej czynności.
\end{itemize}

\newpage

\section{Podsumowanie}
\subsection{Szczegółowe nakłady projektowe członków zespołu}
Tabela (kolumny to osoby, wiersze to działania) pokazująca, kto ile czasu poświęcił na projekt oraz procentowy udział każdej osoby w danym zadaniu oraz wiersz podsumowania – procentowy udział każdej osoby w skali całego projektu

\newpage
\section{Inne informacje}
Przydatne informacje, które nie zostały ujęte we wcześniejszych punktach

\end{document}
